\documentclass[12pt,a4paper]{amsart}

\usepackage[slovene]{babel}
\usepackage[utf8]{inputenc}
\usepackage{amsmath,amssymb,amsfonts}
\usepackage{url}

\textwidth 15cm
\textheight 24cm
\oddsidemargin.5cm
\evensidemargin.5cm
\topmargin-5mm
\addtolength{\footskip}{10pt}
\pagestyle{plain}
\overfullrule=15pt

\theoremstyle{definition}
\newtheorem{definicija}{Definicija}[section]
\newtheorem{primer}[definicija]{Primer}
\newtheorem{opomba}[definicija]{Opomba}

\theoremstyle{plain}
\newtheorem{lema}[definicija]{Lema}
\newtheorem{izrek}[definicija]{Izrek}
\newtheorem{trditev}[definicija]{Trditev}
\newtheorem{posledica}[definicija]{Posledica}

\begin{document}

% Naslovnica
%----------------------------------------------------------------------------------------------------
\thispagestyle{empty}
\noindent{\large
Univerza v Ljubljani\\[1mm]
Fakulteta za matematiko in fiziko\\[3mm]
Finančna matematika -- 1.~stopnja}
\vfill

\begin{center}{\large
Tilen Humar, Urban Rupnik\\[2mm]
{\Huge \bf Iskanje bitonične rešitve problema potujočega trgovca}\\[5mm]
Projekt OR pri predmetu Finančni praktikum\\[1cm]}
\end{center}
\vfill

\noindent{\large
Ljubljana, 2022}
\pagebreak

%----------------------------------------------------------------------------------------------------
\section{Predstavitev problema}

\noindent
{\bf Problem potujočega trgovca} oziroma {\bf problem trgovskega potnika} je ponavadi zastavljen v naslednji obliki.
\newline

\noindent
Obstaja $n$ mest, za katera poznamo razdalje med poljubnim parom mest. 
Trgovec želi obiskati vsa mesta, pri čemer pot začne in konča v istem mestu in vsak kraj obišče natanko enkrat.
Katera je najkrajša oziroma najcenejša pot, ki jo lahko izbere trgovec?
\newline

\noindent
V matematičnem jeziku se problem torej prevede na iskanje najcenejšega Hamiltonovega cikla v polnem grafu $K_n$, kjer ima vsaka povezava $e$ znano utež (ceno) $c_e$. 


\end{document}